% Options for packages loaded elsewhere
\PassOptionsToPackage{unicode}{hyperref}
\PassOptionsToPackage{hyphens}{url}
%
\documentclass[
]{article}
\usepackage{amsmath,amssymb}
\usepackage{lmodern}
\usepackage{iftex}
\ifPDFTeX
  \usepackage[T1]{fontenc}
  \usepackage[utf8]{inputenc}
  \usepackage{textcomp} % provide euro and other symbols
\else % if luatex or xetex
  \usepackage{unicode-math}
  \defaultfontfeatures{Scale=MatchLowercase}
  \defaultfontfeatures[\rmfamily]{Ligatures=TeX,Scale=1}
\fi
% Use upquote if available, for straight quotes in verbatim environments
\IfFileExists{upquote.sty}{\usepackage{upquote}}{}
\IfFileExists{microtype.sty}{% use microtype if available
  \usepackage[]{microtype}
  \UseMicrotypeSet[protrusion]{basicmath} % disable protrusion for tt fonts
}{}
\makeatletter
\@ifundefined{KOMAClassName}{% if non-KOMA class
  \IfFileExists{parskip.sty}{%
    \usepackage{parskip}
  }{% else
    \setlength{\parindent}{0pt}
    \setlength{\parskip}{6pt plus 2pt minus 1pt}}
}{% if KOMA class
  \KOMAoptions{parskip=half}}
\makeatother
\usepackage{xcolor}
\usepackage[margin = 2.54cm]{geometry}
\usepackage{color}
\usepackage{fancyvrb}
\newcommand{\VerbBar}{|}
\newcommand{\VERB}{\Verb[commandchars=\\\{\}]}
\DefineVerbatimEnvironment{Highlighting}{Verbatim}{commandchars=\\\{\}}
% Add ',fontsize=\small' for more characters per line
\usepackage{framed}
\definecolor{shadecolor}{RGB}{248,248,248}
\newenvironment{Shaded}{\begin{snugshade}}{\end{snugshade}}
\newcommand{\AlertTok}[1]{\textcolor[rgb]{0.94,0.16,0.16}{#1}}
\newcommand{\AnnotationTok}[1]{\textcolor[rgb]{0.56,0.35,0.01}{\textbf{\textit{#1}}}}
\newcommand{\AttributeTok}[1]{\textcolor[rgb]{0.77,0.63,0.00}{#1}}
\newcommand{\BaseNTok}[1]{\textcolor[rgb]{0.00,0.00,0.81}{#1}}
\newcommand{\BuiltInTok}[1]{#1}
\newcommand{\CharTok}[1]{\textcolor[rgb]{0.31,0.60,0.02}{#1}}
\newcommand{\CommentTok}[1]{\textcolor[rgb]{0.56,0.35,0.01}{\textit{#1}}}
\newcommand{\CommentVarTok}[1]{\textcolor[rgb]{0.56,0.35,0.01}{\textbf{\textit{#1}}}}
\newcommand{\ConstantTok}[1]{\textcolor[rgb]{0.00,0.00,0.00}{#1}}
\newcommand{\ControlFlowTok}[1]{\textcolor[rgb]{0.13,0.29,0.53}{\textbf{#1}}}
\newcommand{\DataTypeTok}[1]{\textcolor[rgb]{0.13,0.29,0.53}{#1}}
\newcommand{\DecValTok}[1]{\textcolor[rgb]{0.00,0.00,0.81}{#1}}
\newcommand{\DocumentationTok}[1]{\textcolor[rgb]{0.56,0.35,0.01}{\textbf{\textit{#1}}}}
\newcommand{\ErrorTok}[1]{\textcolor[rgb]{0.64,0.00,0.00}{\textbf{#1}}}
\newcommand{\ExtensionTok}[1]{#1}
\newcommand{\FloatTok}[1]{\textcolor[rgb]{0.00,0.00,0.81}{#1}}
\newcommand{\FunctionTok}[1]{\textcolor[rgb]{0.00,0.00,0.00}{#1}}
\newcommand{\ImportTok}[1]{#1}
\newcommand{\InformationTok}[1]{\textcolor[rgb]{0.56,0.35,0.01}{\textbf{\textit{#1}}}}
\newcommand{\KeywordTok}[1]{\textcolor[rgb]{0.13,0.29,0.53}{\textbf{#1}}}
\newcommand{\NormalTok}[1]{#1}
\newcommand{\OperatorTok}[1]{\textcolor[rgb]{0.81,0.36,0.00}{\textbf{#1}}}
\newcommand{\OtherTok}[1]{\textcolor[rgb]{0.56,0.35,0.01}{#1}}
\newcommand{\PreprocessorTok}[1]{\textcolor[rgb]{0.56,0.35,0.01}{\textit{#1}}}
\newcommand{\RegionMarkerTok}[1]{#1}
\newcommand{\SpecialCharTok}[1]{\textcolor[rgb]{0.00,0.00,0.00}{#1}}
\newcommand{\SpecialStringTok}[1]{\textcolor[rgb]{0.31,0.60,0.02}{#1}}
\newcommand{\StringTok}[1]{\textcolor[rgb]{0.31,0.60,0.02}{#1}}
\newcommand{\VariableTok}[1]{\textcolor[rgb]{0.00,0.00,0.00}{#1}}
\newcommand{\VerbatimStringTok}[1]{\textcolor[rgb]{0.31,0.60,0.02}{#1}}
\newcommand{\WarningTok}[1]{\textcolor[rgb]{0.56,0.35,0.01}{\textbf{\textit{#1}}}}
\usepackage{graphicx}
\makeatletter
\def\maxwidth{\ifdim\Gin@nat@width>\linewidth\linewidth\else\Gin@nat@width\fi}
\def\maxheight{\ifdim\Gin@nat@height>\textheight\textheight\else\Gin@nat@height\fi}
\makeatother
% Scale images if necessary, so that they will not overflow the page
% margins by default, and it is still possible to overwrite the defaults
% using explicit options in \includegraphics[width, height, ...]{}
\setkeys{Gin}{width=\maxwidth,height=\maxheight,keepaspectratio}
% Set default figure placement to htbp
\makeatletter
\def\fps@figure{htbp}
\makeatother
\setlength{\emergencystretch}{3em} % prevent overfull lines
\providecommand{\tightlist}{%
  \setlength{\itemsep}{0pt}\setlength{\parskip}{0pt}}
\setcounter{secnumdepth}{-\maxdimen} % remove section numbering
\ifLuaTeX
  \usepackage{selnolig}  % disable illegal ligatures
\fi
\IfFileExists{bookmark.sty}{\usepackage{bookmark}}{\usepackage{hyperref}}
\IfFileExists{xurl.sty}{\usepackage{xurl}}{} % add URL line breaks if available
\urlstyle{same} % disable monospaced font for URLs
\hypersetup{
  pdftitle={QB2023\_Team\_Project: Indiana Pond Bacterial Communities},
  pdfauthor={Erica Nadolski and Joy O'Brien, Z620: Quantitative Biodiversity, Indiana University},
  hidelinks,
  pdfcreator={LaTeX via pandoc}}

\title{QB2023\_Team\_Project: Indiana Pond Bacterial Communities}
\author{Erica Nadolski and Joy O'Brien, Z620: Quantitative Biodiversity,
Indiana University}
\date{02 March, 2023}

\begin{document}
\maketitle

\hypertarget{environment-setup}{%
\subsubsection{environment setup}\label{environment-setup}}

\begin{Shaded}
\begin{Highlighting}[]
\NormalTok{package.list }\OtherTok{\textless{}{-}} \FunctionTok{c}\NormalTok{(}\StringTok{"picante"}\NormalTok{,}\StringTok{\textquotesingle{}ape\textquotesingle{}}\NormalTok{, }\StringTok{\textquotesingle{}seqinr\textquotesingle{}}\NormalTok{, }\StringTok{"vegan"}\NormalTok{,}\StringTok{"fossil"}\NormalTok{,}\StringTok{"reshape"}\NormalTok{,}
                  \StringTok{"devtools"}\NormalTok{,}\StringTok{"BiocManager"}\NormalTok{,}\StringTok{"ineq"}\NormalTok{,}\StringTok{"labdsv"}\NormalTok{,}
                  \StringTok{"matrixStats"}\NormalTok{,}\StringTok{"pROC"}\NormalTok{,}\StringTok{\textquotesingle{}phylobase\textquotesingle{}}\NormalTok{, }\StringTok{\textquotesingle{}adephylo\textquotesingle{}}\NormalTok{, }\StringTok{\textquotesingle{}geiger\textquotesingle{}}\NormalTok{,}
                  \StringTok{\textquotesingle{}stats\textquotesingle{}}\NormalTok{, }\StringTok{\textquotesingle{}RColorBrewer\textquotesingle{}}\NormalTok{, }\StringTok{\textquotesingle{}caper\textquotesingle{}}\NormalTok{, }\StringTok{\textquotesingle{}phylolm\textquotesingle{}}\NormalTok{, }\StringTok{\textquotesingle{}pmc\textquotesingle{}}\NormalTok{, }
                  \StringTok{\textquotesingle{}ggplot2\textquotesingle{}}\NormalTok{, }\StringTok{\textquotesingle{}tidyr\textquotesingle{}}\NormalTok{, }\StringTok{\textquotesingle{}dplyr\textquotesingle{}}\NormalTok{, }\StringTok{\textquotesingle{}phangorn\textquotesingle{}}\NormalTok{, }\StringTok{\textquotesingle{}pander\textquotesingle{}}\NormalTok{, }
                  \StringTok{\textquotesingle{}phylogram\textquotesingle{}}\NormalTok{, }\StringTok{\textquotesingle{}dendextend\textquotesingle{}}\NormalTok{, }\StringTok{"tidyverse"}\NormalTok{, }\StringTok{"ggfortify"}\NormalTok{)}
\ControlFlowTok{for}\NormalTok{ (package }\ControlFlowTok{in}\NormalTok{ package.list) \{}
  \ControlFlowTok{if}\NormalTok{ (}\SpecialCharTok{!}\FunctionTok{require}\NormalTok{(package, }\AttributeTok{character.only=}\ConstantTok{TRUE}\NormalTok{, }\AttributeTok{quietly=}\ConstantTok{TRUE}\NormalTok{)) \{}
    \FunctionTok{install.packages}\NormalTok{(package)}
    \FunctionTok{library}\NormalTok{(package, }\AttributeTok{character.only=}\ConstantTok{TRUE}\NormalTok{)}
\NormalTok{  \}}
\NormalTok{\}}
\end{Highlighting}
\end{Shaded}

\begin{verbatim}
## This is vegan 2.6-4
\end{verbatim}

\begin{verbatim}
## 
## Attaching package: 'seqinr'
\end{verbatim}

\begin{verbatim}
## The following object is masked from 'package:nlme':
## 
##     gls
\end{verbatim}

\begin{verbatim}
## The following object is masked from 'package:permute':
## 
##     getType
\end{verbatim}

\begin{verbatim}
## The following objects are masked from 'package:ape':
## 
##     as.alignment, consensus
\end{verbatim}

\begin{verbatim}
## 
## Attaching package: 'shapefiles'
\end{verbatim}

\begin{verbatim}
## The following objects are masked from 'package:foreign':
## 
##     read.dbf, write.dbf
\end{verbatim}

\begin{verbatim}
## 
## Attaching package: 'devtools'
\end{verbatim}

\begin{verbatim}
## The following object is masked from 'package:permute':
## 
##     check
\end{verbatim}

\begin{verbatim}
## 
## Attaching package: 'BiocManager'
\end{verbatim}

\begin{verbatim}
## The following object is masked from 'package:devtools':
## 
##     install
\end{verbatim}

\begin{verbatim}
## This is mgcv 1.8-41. For overview type 'help("mgcv-package")'.
\end{verbatim}

\begin{verbatim}
## Registered S3 method overwritten by 'labdsv':
##   method       from
##   summary.dist ade4
\end{verbatim}

\begin{verbatim}
## This is labdsv 2.0-1
## convert existing ordinations with as.dsvord()
\end{verbatim}

\begin{verbatim}
## 
## Attaching package: 'labdsv'
\end{verbatim}

\begin{verbatim}
## The following object is masked from 'package:stats':
## 
##     density
\end{verbatim}

\begin{verbatim}
## 
## Attaching package: 'matrixStats'
\end{verbatim}

\begin{verbatim}
## The following object is masked from 'package:seqinr':
## 
##     count
\end{verbatim}

\begin{verbatim}
## Type 'citation("pROC")' for a citation.
\end{verbatim}

\begin{verbatim}
## 
## Attaching package: 'pROC'
\end{verbatim}

\begin{verbatim}
## The following objects are masked from 'package:stats':
## 
##     cov, smooth, var
\end{verbatim}

\begin{verbatim}
## 
## Attaching package: 'phylobase'
\end{verbatim}

\begin{verbatim}
## The following object is masked from 'package:ape':
## 
##     edges
\end{verbatim}

\begin{verbatim}
## 
## Attaching package: 'tidyr'
\end{verbatim}

\begin{verbatim}
## The following objects are masked from 'package:reshape':
## 
##     expand, smiths
\end{verbatim}

\begin{verbatim}
## 
## Attaching package: 'dplyr'
\end{verbatim}

\begin{verbatim}
## The following object is masked from 'package:MASS':
## 
##     select
\end{verbatim}

\begin{verbatim}
## The following object is masked from 'package:matrixStats':
## 
##     count
\end{verbatim}

\begin{verbatim}
## The following object is masked from 'package:reshape':
## 
##     rename
\end{verbatim}

\begin{verbatim}
## The following object is masked from 'package:seqinr':
## 
##     count
\end{verbatim}

\begin{verbatim}
## The following object is masked from 'package:nlme':
## 
##     collapse
\end{verbatim}

\begin{verbatim}
## The following object is masked from 'package:ape':
## 
##     where
\end{verbatim}

\begin{verbatim}
## The following objects are masked from 'package:stats':
## 
##     filter, lag
\end{verbatim}

\begin{verbatim}
## The following objects are masked from 'package:base':
## 
##     intersect, setdiff, setequal, union
\end{verbatim}

\begin{verbatim}
## 
## Attaching package: 'phangorn'
\end{verbatim}

\begin{verbatim}
## The following object is masked from 'package:pROC':
## 
##     coords
\end{verbatim}

\begin{verbatim}
## The following objects are masked from 'package:vegan':
## 
##     diversity, treedist
\end{verbatim}

\begin{verbatim}
## 
## Attaching package: 'phylogram'
\end{verbatim}

\begin{verbatim}
## The following object is masked from 'package:phylobase':
## 
##     prune
\end{verbatim}

\begin{verbatim}
## Registered S3 methods overwritten by 'dendextend':
##   method              from     
##   as.dendrogram.phylo phylogram
##   rev.hclust          vegan
\end{verbatim}

\begin{verbatim}
## 
## ---------------------
## Welcome to dendextend version 1.16.0
## Type citation('dendextend') for how to cite the package.
## 
## Type browseVignettes(package = 'dendextend') for the package vignette.
## The github page is: https://github.com/talgalili/dendextend/
## 
## Suggestions and bug-reports can be submitted at: https://github.com/talgalili/dendextend/issues
## You may ask questions at stackoverflow, use the r and dendextend tags: 
##   https://stackoverflow.com/questions/tagged/dendextend
## 
##  To suppress this message use:  suppressPackageStartupMessages(library(dendextend))
## ---------------------
\end{verbatim}

\begin{verbatim}
## 
## Attaching package: 'dendextend'
\end{verbatim}

\begin{verbatim}
## The following object is masked from 'package:phylogram':
## 
##     prune
\end{verbatim}

\begin{verbatim}
## The following object is masked from 'package:geiger':
## 
##     is.phylo
\end{verbatim}

\begin{verbatim}
## The following objects are masked from 'package:phylobase':
## 
##     labels<-, prune
\end{verbatim}

\begin{verbatim}
## The following object is masked from 'package:permute':
## 
##     shuffle
\end{verbatim}

\begin{verbatim}
## The following objects are masked from 'package:ape':
## 
##     ladderize, rotate
\end{verbatim}

\begin{verbatim}
## The following object is masked from 'package:stats':
## 
##     cutree
\end{verbatim}

\begin{verbatim}
## -- Attaching core tidyverse packages ------------------------ tidyverse 2.0.0 --
## v forcats   1.0.0     v readr     2.1.4
## v lubridate 1.9.2     v stringr   1.5.0
## v purrr     1.0.1     v tibble    3.1.8
## -- Conflicts ------------------------------------------ tidyverse_conflicts() --
## x dplyr::collapse()  masks nlme::collapse()
## x dplyr::count()     masks matrixStats::count(), seqinr::count()
## x tidyr::expand()    masks reshape::expand()
## x dplyr::filter()    masks stats::filter()
## x dplyr::lag()       masks stats::lag()
## x purrr::map()       masks maps::map()
## x dplyr::rename()    masks reshape::rename()
## x dplyr::select()    masks MASS::select()
## x lubridate::stamp() masks reshape::stamp()
## x dplyr::where()     masks ape::where()
## i Use the ]8;;http://conflicted.r-lib.org/conflicted package]8;; to force all conflicts to become errors
\end{verbatim}

\hypertarget{input-and-wrangle-data}{%
\paragraph{Input and wrangle data}\label{input-and-wrangle-data}}

\hypertarget{pca-of-environmental-variables-to-visualize-variation}{%
\subsubsection{PCA of Environmental Variables to visualize
variation}\label{pca-of-environmental-variables-to-visualize-variation}}

\includegraphics{QB2023_Team_Project_files/figure-latex/pressure-1.pdf}

\begin{quote}
\textbf{\emph{Dataset Note}}: We are using a Lennon lab dataset of
microbial DNA and cDNA extracted from local ponds. There are 58 pond
sites, and 34059 species (OTUs) across all the sites. There is high
variance in abundance and evenness across all of the sites; based on
exploratory rank abundance curves, there is high abundance of a few OTUs
and a long tail of low-abundance OTUs.
\end{quote}

\begin{verbatim}

### Alpha Diversity 

```r
# Rank Abundance Curve

RAC <- function(x="")+{
  x.ab = x[x > 0]
  x.ab.ranked = x.ab[order(x.ab, decreasing=TRUE)]
  as.data.frame(lapply(x.ab.ranked, unlist))
  return(x.ab.ranked)
}

# run RAC function
ponds.rac <- as.numeric(RAC(species_mat[1, ]))
length(ponds.rac)
\end{verbatim}

\begin{verbatim}
## [1] 4311
\end{verbatim}

\begin{Shaded}
\begin{Highlighting}[]
\FunctionTok{max}\NormalTok{(ponds.rac)}
\end{Highlighting}
\end{Shaded}

\begin{verbatim}
## [1] 0.1984115
\end{verbatim}

\begin{Shaded}
\begin{Highlighting}[]
\FunctionTok{min}\NormalTok{(ponds.rac)}
\end{Highlighting}
\end{Shaded}

\begin{verbatim}
## [1] 5.844228e-06
\end{verbatim}

\begin{Shaded}
\begin{Highlighting}[]
\FunctionTok{plot.new}\NormalTok{()}
\NormalTok{pond.ranks }\OtherTok{\textless{}{-}} \FunctionTok{as.vector}\NormalTok{(}\FunctionTok{seq}\NormalTok{(}\DecValTok{1}\NormalTok{, }\FunctionTok{length}\NormalTok{(ponds.rac)))}
\NormalTok{opar }\OtherTok{\textless{}{-}} \FunctionTok{par}\NormalTok{(}\AttributeTok{no.readonly =} \ConstantTok{TRUE}\NormalTok{)}
\FunctionTok{par}\NormalTok{(}\AttributeTok{mar =} \FunctionTok{c}\NormalTok{(}\FloatTok{5.1}\NormalTok{, }\FloatTok{5.1}\NormalTok{, }\FloatTok{4.1}\NormalTok{, }\FloatTok{2.1}\NormalTok{))}
\FunctionTok{plot}\NormalTok{(pond.ranks, }\FunctionTok{log}\NormalTok{(ponds.rac), }\AttributeTok{type =} \StringTok{"p"}\NormalTok{, }\AttributeTok{axes =}\NormalTok{F,}
     \AttributeTok{xlab =} \StringTok{"Rank in abundance"}\NormalTok{, }\AttributeTok{ylab =} \StringTok{"Abundance"}\NormalTok{, }
     \AttributeTok{las =} \DecValTok{1}\NormalTok{, }\AttributeTok{cex.lab =} \FloatTok{1.4}\NormalTok{, }\AttributeTok{cex.axis =} \FloatTok{1.25}\NormalTok{);}
\FunctionTok{box}\NormalTok{();}
\FunctionTok{axis}\NormalTok{(}\AttributeTok{side =} \DecValTok{1}\NormalTok{, }\AttributeTok{labels =}\NormalTok{ T, }\AttributeTok{cex.axis =} \FloatTok{1.25}\NormalTok{);}
\FunctionTok{axis}\NormalTok{(}\AttributeTok{side =} \DecValTok{2}\NormalTok{, }\AttributeTok{las =} \DecValTok{1}\NormalTok{, }\AttributeTok{cex.axis =} \FloatTok{1.25}\NormalTok{, }\AttributeTok{labels =} \FunctionTok{c}\NormalTok{(}\DecValTok{1}\NormalTok{, }\DecValTok{10}\NormalTok{, }\DecValTok{100}\NormalTok{, }\DecValTok{1000}\NormalTok{, }\DecValTok{10000}\NormalTok{), }\AttributeTok{at =} \FunctionTok{log}\NormalTok{(}\FunctionTok{c}\NormalTok{(}\DecValTok{1}\NormalTok{, }\DecValTok{10}\NormalTok{, }\DecValTok{100}\NormalTok{, }\DecValTok{1000}\NormalTok{, }\DecValTok{10000}\NormalTok{)))}
\end{Highlighting}
\end{Shaded}

\includegraphics{QB2023_Team_Project_files/figure-latex/unnamed-chunk-3-1.pdf}

\begin{Shaded}
\begin{Highlighting}[]
\DocumentationTok{\#\# create new dataframe }
\NormalTok{alpha\_metrics }\OtherTok{\textless{}{-}} \FunctionTok{data.frame}\NormalTok{(meta\_data[,}\DecValTok{1}\SpecialCharTok{:}\DecValTok{3}\NormalTok{])}

\DocumentationTok{\#\# species richness }
\NormalTok{alpha\_metrics}\SpecialCharTok{$}\NormalTok{richness }\OtherTok{\textless{}{-}} \FunctionTok{specnumber}\NormalTok{(species\_mat)}

\DocumentationTok{\#\# Shannon\textquotesingle{}s diversity}
\NormalTok{alpha\_metrics}\SpecialCharTok{$}\NormalTok{ShanH }\OtherTok{\textless{}{-}}\NormalTok{ vegan}\SpecialCharTok{::}\FunctionTok{diversity}\NormalTok{(species\_mat, }\AttributeTok{index=}\StringTok{"shannon"}\NormalTok{)}

\DocumentationTok{\#\# inverse simpsons}
\NormalTok{alpha\_metrics}\SpecialCharTok{$}\NormalTok{invS }\OtherTok{\textless{}{-}}\NormalTok{ vegan}\SpecialCharTok{::}\FunctionTok{diversity}\NormalTok{(species\_mat, }\AttributeTok{index=}\StringTok{"invsimpson"}\NormalTok{)}

\FunctionTok{ggplot}\NormalTok{(alpha\_metrics, }\FunctionTok{aes}\NormalTok{(}\AttributeTok{x=}\NormalTok{source, }\AttributeTok{y=}\NormalTok{richness, }\AttributeTok{fill=}\NormalTok{Location))}\SpecialCharTok{+} 
  \FunctionTok{geom\_boxplot}\NormalTok{()}\SpecialCharTok{+}
  \FunctionTok{labs}\NormalTok{(}\AttributeTok{title=}\StringTok{"Indiana Pond Species Richness"}\NormalTok{,}\AttributeTok{x=}\StringTok{"Source Genetic Material"}\NormalTok{, }\AttributeTok{y =} \StringTok{"Richness"}\NormalTok{)}\SpecialCharTok{+}
  \FunctionTok{geom\_dotplot}\NormalTok{(}\AttributeTok{binaxis=}\StringTok{\textquotesingle{}y\textquotesingle{}}\NormalTok{, }\AttributeTok{stackdir=}\StringTok{\textquotesingle{}center\textquotesingle{}}\NormalTok{, }\AttributeTok{dotsize=}\FloatTok{0.5}\NormalTok{, }\AttributeTok{position=}\FunctionTok{position\_dodge}\NormalTok{(}\FloatTok{0.75}\NormalTok{))}\SpecialCharTok{+}
  \FunctionTok{theme\_classic}\NormalTok{()}
\end{Highlighting}
\end{Shaded}

\begin{verbatim}
## Bin width defaults to 1/30 of the range of the data. Pick better value with
## `binwidth`.
\end{verbatim}

\includegraphics{QB2023_Team_Project_files/figure-latex/unnamed-chunk-3-2.pdf}

\begin{Shaded}
\begin{Highlighting}[]
\FunctionTok{ggplot}\NormalTok{(alpha\_metrics, }\FunctionTok{aes}\NormalTok{(}\AttributeTok{x=}\NormalTok{source, }\AttributeTok{y=}\NormalTok{ShanH, }\AttributeTok{fill=}\NormalTok{Location))}\SpecialCharTok{+} 
  \FunctionTok{geom\_boxplot}\NormalTok{()}\SpecialCharTok{+}
  \FunctionTok{labs}\NormalTok{(}\AttributeTok{title=}\StringTok{"Indiana Pond Shannon Diversity"}\NormalTok{,}\AttributeTok{x=}\StringTok{"Source Genetic Material"}\NormalTok{, }\AttributeTok{y =} \StringTok{"Shannon\textquotesingle{}s H"}\NormalTok{)}\SpecialCharTok{+}
  \FunctionTok{geom\_dotplot}\NormalTok{(}\AttributeTok{binaxis=}\StringTok{\textquotesingle{}y\textquotesingle{}}\NormalTok{, }\AttributeTok{stackdir=}\StringTok{\textquotesingle{}center\textquotesingle{}}\NormalTok{, }\AttributeTok{dotsize=}\FloatTok{0.5}\NormalTok{, }\AttributeTok{position=}\FunctionTok{position\_dodge}\NormalTok{(}\FloatTok{0.75}\NormalTok{))}\SpecialCharTok{+}
  \FunctionTok{theme\_classic}\NormalTok{()}
\end{Highlighting}
\end{Shaded}

\begin{verbatim}
## Bin width defaults to 1/30 of the range of the data. Pick better value with
## `binwidth`.
\end{verbatim}

\includegraphics{QB2023_Team_Project_files/figure-latex/unnamed-chunk-3-3.pdf}

\begin{Shaded}
\begin{Highlighting}[]
\FunctionTok{ggplot}\NormalTok{(alpha\_metrics, }\FunctionTok{aes}\NormalTok{(}\AttributeTok{x=}\NormalTok{source, }\AttributeTok{y=}\NormalTok{invS, }\AttributeTok{fill=}\NormalTok{Location))}\SpecialCharTok{+} 
  \FunctionTok{geom\_boxplot}\NormalTok{()}\SpecialCharTok{+}
  \FunctionTok{labs}\NormalTok{(}\AttributeTok{title=}\StringTok{"Indiana Pond Inverse Simpson"}\NormalTok{,}\AttributeTok{x=}\StringTok{"Source Genetic Material"}\NormalTok{, }\AttributeTok{y =} \StringTok{"Inverse Simpson"}\NormalTok{)}\SpecialCharTok{+}
  \FunctionTok{geom\_dotplot}\NormalTok{(}\AttributeTok{binaxis=}\StringTok{\textquotesingle{}y\textquotesingle{}}\NormalTok{, }\AttributeTok{stackdir=}\StringTok{\textquotesingle{}center\textquotesingle{}}\NormalTok{, }\AttributeTok{dotsize=}\FloatTok{0.5}\NormalTok{, }\AttributeTok{position=}\FunctionTok{position\_dodge}\NormalTok{(}\FloatTok{0.75}\NormalTok{))}\SpecialCharTok{+}
  \FunctionTok{theme\_classic}\NormalTok{()}
\end{Highlighting}
\end{Shaded}

\begin{verbatim}
## Bin width defaults to 1/30 of the range of the data. Pick better value with
## `binwidth`.
\end{verbatim}

\includegraphics{QB2023_Team_Project_files/figure-latex/unnamed-chunk-3-4.pdf}

\begin{Shaded}
\begin{Highlighting}[]
\FunctionTok{mean}\NormalTok{(alpha\_metrics}\SpecialCharTok{$}\NormalTok{richness)}
\end{Highlighting}
\end{Shaded}

\begin{verbatim}
## [1] 1975.336
\end{verbatim}

\begin{Shaded}
\begin{Highlighting}[]
\FunctionTok{mean}\NormalTok{(alpha\_metrics}\SpecialCharTok{$}\NormalTok{ShanH)}
\end{Highlighting}
\end{Shaded}

\begin{verbatim}
## [1] 3.758241
\end{verbatim}

\begin{Shaded}
\begin{Highlighting}[]
\FunctionTok{mean}\NormalTok{(alpha\_metrics}\SpecialCharTok{$}\NormalTok{invS)}
\end{Highlighting}
\end{Shaded}

\begin{verbatim}
## [1] 17.75242
\end{verbatim}

\hypertarget{beta-diversity---visualization}{%
\subsubsection{Beta Diversity -
Visualization}\label{beta-diversity---visualization}}

\begin{Shaded}
\begin{Highlighting}[]
\FunctionTok{library}\NormalTok{(viridis)}
\end{Highlighting}
\end{Shaded}

\begin{verbatim}
## Loading required package: viridisLite
\end{verbatim}

\begin{verbatim}
## 
## Attaching package: 'viridis'
\end{verbatim}

\begin{verbatim}
## The following object is masked from 'package:maps':
## 
##     unemp
\end{verbatim}

\begin{Shaded}
\begin{Highlighting}[]
\CommentTok{\# Bray Curtis resemblance matrix}
\NormalTok{total.db }\OtherTok{\textless{}{-}} \FunctionTok{vegdist}\NormalTok{(species\_mat, }\AttributeTok{method=}\StringTok{"bray"}\NormalTok{)}

\CommentTok{\# Heatmap }
\FunctionTok{levelplot}\NormalTok{(}\FunctionTok{as.matrix}\NormalTok{(total.db), }\AttributeTok{aspect=}\StringTok{"iso"}\NormalTok{, }\AttributeTok{col.regions=}\NormalTok{inferno,}
          \AttributeTok{xlab=}\StringTok{"Pond Site"}\NormalTok{, }\AttributeTok{ylab=} \StringTok{"Pond Site"}\NormalTok{, }\AttributeTok{scales=}\FunctionTok{list}\NormalTok{(}\AttributeTok{cex=}\FloatTok{0.5}\NormalTok{), }
          \AttributeTok{main=} \StringTok{"Bray{-}Curtis Distance"}\NormalTok{)}
\end{Highlighting}
\end{Shaded}

\includegraphics{QB2023_Team_Project_files/figure-latex/unnamed-chunk-4-1.pdf}

\begin{Shaded}
\begin{Highlighting}[]
\CommentTok{\# Wards cluster analysis}
\NormalTok{total.ward }\OtherTok{\textless{}{-}} \FunctionTok{hclust}\NormalTok{(total.db, }\AttributeTok{method=} \StringTok{"ward.D2"}\NormalTok{)}

\FunctionTok{par}\NormalTok{(}\AttributeTok{mar =} \FunctionTok{c}\NormalTok{(}\DecValTok{1}\NormalTok{, }\DecValTok{5}\NormalTok{, }\DecValTok{2}\NormalTok{, }\DecValTok{2}\NormalTok{) }\SpecialCharTok{+} \FloatTok{0.1}\NormalTok{)}
\FunctionTok{plot}\NormalTok{(total.ward, }\AttributeTok{main=}\StringTok{"Indiana Pond Bacteria: Ward\textquotesingle{}s Clustering"}\NormalTok{, }\AttributeTok{ylab=} \StringTok{"Squared Bray{-}Curtis Distance"}\NormalTok{)}
\end{Highlighting}
\end{Shaded}

\includegraphics{QB2023_Team_Project_files/figure-latex/unnamed-chunk-4-2.pdf}

\begin{Shaded}
\begin{Highlighting}[]
\CommentTok{\# Principal Component Analysis}
\NormalTok{total.pcoa }\OtherTok{\textless{}{-}} \FunctionTok{cmdscale}\NormalTok{(total.db, }\AttributeTok{eig=}\ConstantTok{TRUE}\NormalTok{, }\AttributeTok{k=}\DecValTok{3}\NormalTok{)}

\NormalTok{exvar1 }\OtherTok{\textless{}{-}} \FunctionTok{round}\NormalTok{(total.pcoa}\SpecialCharTok{$}\NormalTok{eig[}\DecValTok{1}\NormalTok{] }\SpecialCharTok{/} \FunctionTok{sum}\NormalTok{(total.pcoa}\SpecialCharTok{$}\NormalTok{eig), }\DecValTok{3}\NormalTok{) }\SpecialCharTok{*} \DecValTok{100}
\NormalTok{exvar2 }\OtherTok{\textless{}{-}} \FunctionTok{round}\NormalTok{(total.pcoa}\SpecialCharTok{$}\NormalTok{eig[}\DecValTok{2}\NormalTok{] }\SpecialCharTok{/} \FunctionTok{sum}\NormalTok{(total.pcoa}\SpecialCharTok{$}\NormalTok{eig), }\DecValTok{3}\NormalTok{) }\SpecialCharTok{*} \DecValTok{100}
\NormalTok{exvar3 }\OtherTok{\textless{}{-}} \FunctionTok{round}\NormalTok{(total.pcoa}\SpecialCharTok{$}\NormalTok{eig[}\DecValTok{3}\NormalTok{] }\SpecialCharTok{/} \FunctionTok{sum}\NormalTok{(total.pcoa}\SpecialCharTok{$}\NormalTok{eig), }\DecValTok{3}\NormalTok{) }\SpecialCharTok{*} \DecValTok{100}
\NormalTok{total.sum.eig }\OtherTok{\textless{}{-}} \FunctionTok{sum}\NormalTok{(exvar1, exvar2, exvar3)}

\CommentTok{\# PCoA Plot PC1 x PC2 colored by state park}
\FunctionTok{par}\NormalTok{(}\AttributeTok{mar =} \FunctionTok{c}\NormalTok{(}\DecValTok{2}\NormalTok{, }\DecValTok{2}\NormalTok{, }\DecValTok{2}\NormalTok{, }\DecValTok{2}\NormalTok{) }\SpecialCharTok{+} \DecValTok{3}\NormalTok{)}
\FunctionTok{plot}\NormalTok{(total.pcoa}\SpecialCharTok{$}\NormalTok{points[ ,}\DecValTok{1}\NormalTok{], total.pcoa}\SpecialCharTok{$}\NormalTok{points[ ,}\DecValTok{2}\NormalTok{],}
     \CommentTok{\#xlim = c({-}0.55, {-}0.3),}
     \CommentTok{\#ylim = c({-}0.03, 0.01),}
     \AttributeTok{xlab=} \FunctionTok{paste}\NormalTok{(}\StringTok{"PCoA 1 ("}\NormalTok{, exvar1, }\StringTok{"\%)"}\NormalTok{, }\AttributeTok{sep =} \StringTok{""}\NormalTok{),}
     \AttributeTok{ylab=} \FunctionTok{paste}\NormalTok{(}\StringTok{"PCoA 2 ("}\NormalTok{, exvar2, }\StringTok{"\%)"}\NormalTok{, }\AttributeTok{sep =} \StringTok{""}\NormalTok{),}
     \AttributeTok{pch =} \DecValTok{16}\NormalTok{, }\AttributeTok{cex =} \FloatTok{2.0}\NormalTok{, }\AttributeTok{type =} \StringTok{"n"}\NormalTok{, }\AttributeTok{cex.lab =} \FloatTok{1.5}\NormalTok{,}
     \AttributeTok{cex.axis=}\FloatTok{1.2}\NormalTok{, }\AttributeTok{axes=}\ConstantTok{FALSE}\NormalTok{);}
\FunctionTok{axis}\NormalTok{(}\AttributeTok{side =} \DecValTok{1}\NormalTok{, }\AttributeTok{labels =}\NormalTok{ T, }\AttributeTok{lwd.ticks =} \DecValTok{2}\NormalTok{, }\AttributeTok{cex.axis =} \FloatTok{1.2}\NormalTok{, }\AttributeTok{las =} \DecValTok{1}\NormalTok{);}
\FunctionTok{axis}\NormalTok{(}\AttributeTok{side =} \DecValTok{2}\NormalTok{, }\AttributeTok{labels =}\NormalTok{ T, }\AttributeTok{lwd.ticks =} \DecValTok{2}\NormalTok{, }\AttributeTok{cex.axis =} \FloatTok{1.2}\NormalTok{, }\AttributeTok{las =} \DecValTok{1}\NormalTok{);}
\FunctionTok{abline}\NormalTok{(}\AttributeTok{h =} \DecValTok{0}\NormalTok{, }\AttributeTok{v =} \DecValTok{0}\NormalTok{, }\AttributeTok{lty =} \DecValTok{3}\NormalTok{);}
\FunctionTok{box}\NormalTok{(}\AttributeTok{lwd =} \DecValTok{2}\NormalTok{);                                        }
\FunctionTok{points}\NormalTok{(total.pcoa}\SpecialCharTok{$}\NormalTok{points[}\DecValTok{1}\SpecialCharTok{:}\DecValTok{18}\NormalTok{,}\DecValTok{1}\NormalTok{], total.pcoa}\SpecialCharTok{$}\NormalTok{points[}\DecValTok{1}\SpecialCharTok{:}\DecValTok{18}\NormalTok{,}\DecValTok{2}\NormalTok{],}
       \AttributeTok{pch =} \DecValTok{20}\NormalTok{, }\AttributeTok{cex =} \DecValTok{2}\NormalTok{, }\AttributeTok{bg =} \StringTok{"red"}\NormalTok{, }\AttributeTok{col =} \StringTok{"red"}\NormalTok{); }\CommentTok{\# BC}
\FunctionTok{points}\NormalTok{(total.pcoa}\SpecialCharTok{$}\NormalTok{points[}\DecValTok{19}\SpecialCharTok{:}\DecValTok{41}\NormalTok{,}\DecValTok{1}\NormalTok{], total.pcoa}\SpecialCharTok{$}\NormalTok{points[}\DecValTok{19}\SpecialCharTok{:}\DecValTok{41}\NormalTok{,}\DecValTok{2}\NormalTok{],}
       \AttributeTok{pch =} \DecValTok{20}\NormalTok{, }\AttributeTok{cex =} \DecValTok{2}\NormalTok{, }\AttributeTok{bg =} \StringTok{"green"}\NormalTok{, }\AttributeTok{col =} \StringTok{"green"}\NormalTok{); }\CommentTok{\# H}
\FunctionTok{points}\NormalTok{(total.pcoa}\SpecialCharTok{$}\NormalTok{points[}\DecValTok{42}\SpecialCharTok{:}\DecValTok{58}\NormalTok{,}\DecValTok{1}\NormalTok{], total.pcoa}\SpecialCharTok{$}\NormalTok{points[}\DecValTok{42}\SpecialCharTok{:}\DecValTok{58}\NormalTok{,}\DecValTok{2}\NormalTok{],}
       \AttributeTok{pch =} \DecValTok{20}\NormalTok{, }\AttributeTok{cex =} \DecValTok{2}\NormalTok{, }\AttributeTok{bg =} \StringTok{"blue"}\NormalTok{, }\AttributeTok{col =} \StringTok{"blue"}\NormalTok{);}\CommentTok{\# YW}
\FunctionTok{legend}\NormalTok{(}\AttributeTok{x=}\StringTok{"topleft"}\NormalTok{,,}\AttributeTok{legend=}\FunctionTok{c}\NormalTok{(}\StringTok{"Brown County"}\NormalTok{, }\StringTok{"Hoosier Natl. Forest"}\NormalTok{, }\StringTok{"Yellowood"}\NormalTok{),}
       \AttributeTok{fill=}\FunctionTok{c}\NormalTok{(}\StringTok{"red"}\NormalTok{,}\StringTok{"green"}\NormalTok{,}\StringTok{"blue"}\NormalTok{))}
\end{Highlighting}
\end{Shaded}

\includegraphics{QB2023_Team_Project_files/figure-latex/unnamed-chunk-4-3.pdf}

\hypertarget{beta-diversity---hypothesis-testing}{%
\subsubsection{Beta Diversity - Hypothesis
Testing}\label{beta-diversity---hypothesis-testing}}

\hypertarget{mantel-test}{%
\subsubsection{Mantel Test}\label{mantel-test}}

\begin{Shaded}
\begin{Highlighting}[]
\CommentTok{\# Mantel test to test the hypothesis that pond assemblages are correlated with pond environmental variables.}
\CommentTok{\# define matrices}
\CommentTok{\# input right matrices }
\NormalTok{OTU.dist }\OtherTok{\textless{}{-}} \FunctionTok{vegdist}\NormalTok{(species\_mat, }\AttributeTok{method=}\StringTok{"bray"}\NormalTok{)}
\NormalTok{pond.env.dist }\OtherTok{\textless{}{-}} \FunctionTok{vegdist}\NormalTok{(env\_data, }\AttributeTok{method=}\StringTok{"euclid"}\NormalTok{, }\AttributeTok{na.rm =} \ConstantTok{TRUE}\NormalTok{)}

\CommentTok{\# mantel test}
\FunctionTok{mantel}\NormalTok{(OTU.dist,pond.env.dist)}
\end{Highlighting}
\end{Shaded}

\begin{verbatim}
## 
## Mantel statistic based on Pearson's product-moment correlation 
## 
## Call:
## mantel(xdis = OTU.dist, ydis = pond.env.dist) 
## 
## Mantel statistic r: 0.2299 
##       Significance: 0.001 
## 
## Upper quantiles of permutations (null model):
##    90%    95%  97.5%    99% 
## 0.0597 0.0815 0.0948 0.1219 
## Permutation: free
## Number of permutations: 999
\end{verbatim}

\hypertarget{hypothesis-testing---nonmetric-multidimensional-scaling-nmds}{%
\subsubsection{hypothesis testing - nonmetric multidimensional scaling
NMDS}\label{hypothesis-testing---nonmetric-multidimensional-scaling-nmds}}

\begin{Shaded}
\begin{Highlighting}[]
\CommentTok{\#NMDS ordination }
\FunctionTok{set.seed}\NormalTok{(}\DecValTok{123456}\NormalTok{) }\CommentTok{\#set seed is to fix the random number generator so you get the same results }
\CommentTok{\#each time when you run the tests }
\NormalTok{nmds }\OtherTok{\textless{}{-}} \FunctionTok{metaMDS}\NormalTok{(species\_mat, }\AttributeTok{distance =} \StringTok{"bray"}\NormalTok{) }
\end{Highlighting}
\end{Shaded}

\begin{verbatim}
## Run 0 stress 0.2181188 
## Run 1 stress 0.219863 
## Run 2 stress 0.2132913 
## ... New best solution
## ... Procrustes: rmse 0.05581722  max resid 0.2077347 
## Run 3 stress 0.216341 
## Run 4 stress 0.237245 
## Run 5 stress 0.2211872 
## Run 6 stress 0.2159201 
## Run 7 stress 0.2344659 
## Run 8 stress 0.2140572 
## Run 9 stress 0.2161477 
## Run 10 stress 0.2153479 
## Run 11 stress 0.2275295 
## Run 12 stress 0.2195643 
## Run 13 stress 0.2195578 
## Run 14 stress 0.2178924 
## Run 15 stress 0.2182254 
## Run 16 stress 0.235312 
## Run 17 stress 0.2181086 
## Run 18 stress 0.2283044 
## Run 19 stress 0.2353372 
## Run 20 stress 0.2276469 
## *** Best solution was not repeated -- monoMDS stopping criteria:
##     20: stress ratio > sratmax
\end{verbatim}

\begin{Shaded}
\begin{Highlighting}[]
\CommentTok{\# extract data from the ordination to plot }
\CommentTok{\# site scores}
\NormalTok{data.scores          }\OtherTok{\textless{}{-}} \FunctionTok{as.data.frame}\NormalTok{(}\FunctionTok{scores}\NormalTok{(nmds)}\SpecialCharTok{$}\NormalTok{sites)  }
\NormalTok{data.scores}\SpecialCharTok{$}\NormalTok{site     }\OtherTok{\textless{}{-}}\NormalTok{ allData}\SpecialCharTok{$}\NormalTok{site   }
\NormalTok{data.scores}\SpecialCharTok{$}\NormalTok{source   }\OtherTok{\textless{}{-}}\NormalTok{ allData}\SpecialCharTok{$}\NormalTok{source}
\NormalTok{data.scores}\SpecialCharTok{$}\NormalTok{location }\OtherTok{\textless{}{-}}\NormalTok{ allData}\SpecialCharTok{$}\NormalTok{Location }

\CommentTok{\# species scores technically we can also test species significance if we want to }
\NormalTok{species.scores         }\OtherTok{\textless{}{-}} \FunctionTok{as.data.frame}\NormalTok{(}\FunctionTok{scores}\NormalTok{(nmds, }\StringTok{"species"}\NormalTok{))  }
\NormalTok{species.scores}\SpecialCharTok{$}\NormalTok{species }\OtherTok{\textless{}{-}} \FunctionTok{rownames}\NormalTok{(species.scores)  }

\CommentTok{\# filter species ordination scores with the most abundant species }
\NormalTok{species.scores.abundant }\OtherTok{\textless{}{-}} \FunctionTok{filter}\NormalTok{(species.scores, species }\SpecialCharTok{\%in\%}\NormalTok{ selected) }

\CommentTok{\# shortening names}
\NormalTok{species.scores.abundant}\SpecialCharTok{$}\NormalTok{name    }\OtherTok{\textless{}{-}} \FunctionTok{gsub}\NormalTok{(}\StringTok{"000*"}\NormalTok{, }\StringTok{""}\NormalTok{ , }\FunctionTok{rownames}\NormalTok{(species.scores.abundant))}

\CommentTok{\# permutation tests with factors (RNA/DNA and location )}
\NormalTok{distance  }\OtherTok{\textless{}{-}} \FunctionTok{vegdist}\NormalTok{(species\_mat, }\StringTok{"bray"}\NormalTok{)}
\FunctionTok{set.seed}\NormalTok{(}\DecValTok{42}\NormalTok{)}
\NormalTok{permanova }\OtherTok{\textless{}{-}} \FunctionTok{adonis2}\NormalTok{(distance }\SpecialCharTok{\textasciitilde{}}\NormalTok{ . , }\AttributeTok{data =}\NormalTok{ meta\_data[}\DecValTok{2}\SpecialCharTok{:}\DecValTok{3}\NormalTok{], }\AttributeTok{permutations =} \DecValTok{999}\NormalTok{, }\AttributeTok{na.action =}\NormalTok{ na.omit)}
\NormalTok{permanova }\CommentTok{\# both are significant }
\end{Highlighting}
\end{Shaded}

\begin{verbatim}
## Permutation test for adonis under reduced model
## Terms added sequentially (first to last)
## Permutation: free
## Number of permutations: 999
## 
## adonis2(formula = distance ~ ., data = meta_data[2:3], permutations = 999, na.action = na.omit)
##           Df SumOfSqs      R2      F Pr(>F)    
## source     1   1.7358 0.06857 8.9880  0.001 ***
## Location   2   2.5293 0.09991 6.5484  0.001 ***
## Residual 109  21.0504 0.83152                  
## Total    112  25.3154 1.00000                  
## ---
## Signif. codes:  0 '***' 0.001 '**' 0.01 '*' 0.05 '.' 0.1 ' ' 1
\end{verbatim}

\begin{Shaded}
\begin{Highlighting}[]
\CommentTok{\# environmental vectors (envfit only works with continuous )}
\FunctionTok{set.seed}\NormalTok{(}\DecValTok{55}\NormalTok{)}
\NormalTok{fit        }\OtherTok{\textless{}{-}} \FunctionTok{envfit}\NormalTok{(nmds, env\_scaled, }\AttributeTok{na.rm =} \ConstantTok{TRUE}\NormalTok{, }\AttributeTok{permutations =} \DecValTok{999}\NormalTok{)}

\CommentTok{\# vector data of enviromental fit}
\NormalTok{arrows     }\OtherTok{\textless{}{-}} \FunctionTok{data.frame}\NormalTok{(fit}\SpecialCharTok{$}\NormalTok{vector}\SpecialCharTok{$}\NormalTok{arrows, }\AttributeTok{R =}\NormalTok{ fit}\SpecialCharTok{$}\NormalTok{vectors}\SpecialCharTok{$}\NormalTok{r, }\AttributeTok{P =}\NormalTok{ fit}\SpecialCharTok{$}\NormalTok{vectors}\SpecialCharTok{$}\NormalTok{pvals)}
\NormalTok{arrows}\SpecialCharTok{$}\NormalTok{env }\OtherTok{\textless{}{-}} \FunctionTok{rownames}\NormalTok{(arrows)}
\NormalTok{arrows.p   }\OtherTok{\textless{}{-}}\NormalTok{ arrows[arrows}\SpecialCharTok{$}\NormalTok{P }\SpecialCharTok{\textless{}} \FloatTok{0.05}\NormalTok{,] }\CommentTok{\#select the significant variables }
\NormalTok{arrows.p}
\end{Highlighting}
\end{Shaded}

\begin{verbatim}
##                 NMDS1       NMDS2         R     P        env
## lat        -0.9866859 -0.16263727 0.2945887 0.001        lat
## long       -0.8670526 -0.49821668 0.1125202 0.003       long
## Diameter    0.9828689  0.18430597 0.3216353 0.001   Diameter
## Depth      -0.6678566 -0.74428997 0.1286984 0.002      Depth
## Cal_Volume  0.9691860  0.24632994 0.3215025 0.001 Cal_Volume
## ORP         0.8885500  0.45877985 0.1948272 0.001        ORP
## Temp        0.9774520  0.21115751 0.2868514 0.001       Temp
## SpC        -0.7712245 -0.63656328 0.2429217 0.001        SpC
## DO          0.9226521  0.38563345 0.2382033 0.001         DO
## TDS        -0.7523184 -0.65879968 0.3161557 0.001        TDS
## Salinity   -0.7511047 -0.66018310 0.3208778 0.001   Salinity
## pH         -0.4465827 -0.89474237 0.1166368 0.004         pH
## Color      -0.9484831 -0.31682760 0.1061052 0.005      Color
## DON        -0.4176273 -0.90861841 0.1237157 0.003        DON
## canopy      0.9240390  0.38229833 0.4778967 0.001     canopy
## TP         -0.9987446 -0.05009149 0.1751967 0.001         TP
\end{verbatim}

\begin{Shaded}
\begin{Highlighting}[]
\CommentTok{\# plot }
\FunctionTok{ggplot}\NormalTok{()}\SpecialCharTok{+} 
  \FunctionTok{geom\_hline}\NormalTok{(}\AttributeTok{yintercept =} \DecValTok{0}\NormalTok{, }\AttributeTok{linetype =} \StringTok{"dashed"}\NormalTok{)}\SpecialCharTok{+}
  \FunctionTok{geom\_vline}\NormalTok{(}\AttributeTok{xintercept =} \DecValTok{0}\NormalTok{, }\AttributeTok{linetype =} \StringTok{"dashed"}\NormalTok{)}\SpecialCharTok{+}
  \FunctionTok{geom\_point}\NormalTok{(}\AttributeTok{data =}\NormalTok{ data.scores, }\AttributeTok{mapping =} \FunctionTok{aes}\NormalTok{(}\AttributeTok{x =}\NormalTok{ NMDS1, }\AttributeTok{y =}\NormalTok{ NMDS2, }\AttributeTok{colour =}\NormalTok{ location, }\AttributeTok{shape =}\NormalTok{ source), }\AttributeTok{size=}\DecValTok{3}\NormalTok{, }\AttributeTok{alpha =}\NormalTok{ .}\DecValTok{6}\NormalTok{) }\SpecialCharTok{+}
  \FunctionTok{theme\_bw}\NormalTok{()}\SpecialCharTok{+}
  \FunctionTok{coord\_equal}\NormalTok{()}\SpecialCharTok{+}
  \FunctionTok{geom\_segment}\NormalTok{(}\AttributeTok{data =}\NormalTok{ arrows.p, }\FunctionTok{aes}\NormalTok{(}\AttributeTok{x =} \DecValTok{0}\NormalTok{, }\AttributeTok{y =} \DecValTok{0}\NormalTok{, }\AttributeTok{xend =}\NormalTok{ NMDS1, }\AttributeTok{yend =}\NormalTok{ NMDS2), }
               \AttributeTok{arrow =} \FunctionTok{arrow}\NormalTok{(}\AttributeTok{length =} \FunctionTok{unit}\NormalTok{(.}\DecValTok{2}\NormalTok{, }\StringTok{"cm"}\NormalTok{)}\SpecialCharTok{*}\NormalTok{arrows.p}\SpecialCharTok{$}\NormalTok{R),  }\AttributeTok{color =} \StringTok{"grey"}\NormalTok{)}\SpecialCharTok{+}
  \FunctionTok{geom\_text}\NormalTok{(}\AttributeTok{data =}\NormalTok{ arrows.p, }\FunctionTok{aes}\NormalTok{(}\AttributeTok{x =}\NormalTok{ NMDS1, }\AttributeTok{y =}\NormalTok{ NMDS2, }\AttributeTok{label =}\NormalTok{ env), }\AttributeTok{size=}\DecValTok{5}\NormalTok{)}\SpecialCharTok{+}
  \FunctionTok{geom\_text}\NormalTok{(}\AttributeTok{data =}\NormalTok{ species.scores.abundant, }\FunctionTok{aes}\NormalTok{(}\AttributeTok{x =}\NormalTok{ NMDS1, }\AttributeTok{y =}\NormalTok{ NMDS2, }\AttributeTok{label =}\NormalTok{ name), }\AttributeTok{size =} \DecValTok{3}\NormalTok{)}\CommentTok{\# add abundant species scores}
\end{Highlighting}
\end{Shaded}

\includegraphics{QB2023_Team_Project_files/figure-latex/unnamed-chunk-6-1.pdf}
\textgreater{} \textbf{\emph{Ordination Description}}: It appears that
the BC and HNF sites cluster more closely together into one cluster, and
the YSF sites cluster more tightly into another cluster. According to
the Mantel test, 23\% of the variation in pond community similarity
varies in correspondence with pond environmental similarity. The NMDS
with envfit also shows that a variety of environmental variables covary
with community composition along the first two NMDS axes, as well as a
number of indicator species.

\end{document}
